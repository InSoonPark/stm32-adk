\section{Introduction}
\label{intro}

During the Google I/O 2011 conference, Google announced the Android Open Accessory APIs for Android. These APIs allow USB accessories to connect to Android devices running Android 3.1 or Android 2.3.4 without special licensing or fees. The new ``Accessory mode'' does not require the Android device to support USB Host mode. In accessory mode the Android phone or tablet acts as the USB Device and the accessory acts as the USB Host. Hence, Android Open ADK opens up the possibility to easily realize a set of new accessories fully compatible with Android-powered devices that do not support USB Host functionality. For this reason, several distributors started to produce Android Open Accessory compatible development boards, in order to promote the diffusion of this type of accessories. The aim of this work is to add the support of the Android Open ADK to STM32 microcontrollers' family and to develop a demo application in order to show how this protocol can be used. The iNEMOv2 board has been chosen as the target platform for this test case: the goal is to realize the porting of the Windows application iNEMO Suite to Android.

The document is organized in six sections. After this short introduction, an overview on the hardware setup is provided in Section \ref{hw_sec}. Section \ref{adk_sec} describes the main features of the Android Accessory protocol. The Communication Protocol adopted is described in Section \ref{cp_sec}. Section \ref{sw_sec} is focused on the descrition of the key points of the firmware and of the Android application developed. Finally Section \ref{conclusions} discusses the obtained results and the next steps that will be achieved in future.

Source code and documentation for this work are freely accessible from the GIT repository located at \url{http://code.google.com/p/stm32-adk/}. 
